\documentclass[a4paper]{report} %padrao letterpaper, 10pt
\usepackage[utf8]{inputenc}
\usepackage[brazil]{babel}
\usepackage{amsfonts,amssymb,graphicx,enumerate}
\usepackage[centertags]{amsmath}
% Configuracoes de pagina
\usepackage[lmargin=3cm,rmargin=3cm,tmargin=3cm,bmargin=3cm]{geometry}
% Layout da pagina
\usepackage{hyperref}
% Layout para código
\usepackage{listings}
% Font for code style
\usepackage{courier}
% The 68 standard colors known to dvips
\usepackage[usenames,dvipsnames]{xcolor}
\hypersetup{pdfpagelayout=SinglePage, % ou TwoPageLeft
    colorlinks=true,
    pdftitle={Relatório de asuhaush Redes},
    pdfauthor={Áulus Diniz, Arthur Jahn, Gabriel Araújo, Jonatas Lenon}}

%*******************************************************
\definecolor{light-gray}{gray}{0.90}
\lstset{
  backgroundcolor=\color{light-gray},
  basicstyle=\footnotesize\ttfamily,
  breaklines=true
}

\title{Fundamentos de Redes de Computadores: \\ Servidor DNS com Extensões DNSSEC}
\author{Arthur Jahn, Áulus Diniz, Gabriel Araújo, Jonatas Lenon}
\date{15 de Novembro de 2015}    %para ocultar a data digite: \date{ }
%*******************************************************
\begin{document}    %Inicio do documento
\maketitle  %cria o titulo na capa

\tableofcontents %Sumario
%-------------------------------------------------------
\chapter{Problema e Motivação}
\label{chap_problema}

O DNS provê serviços de resolução de \textit{hosts} por meio da tradução de nomes de domínios em números de endereçamento IP. Como um mesmo domínio pode estar vinculado a vários endereços IP, esse serviço pode ser responsável também por distribuição de carga entre os \textit{hosts} de destino, alternando o endereço fornecido ao cliente quando uma solicitação de resolução de domínio é feita. 

Quando o DNS foi projetado, no início dos anos 80, não foram pensadas questões de segurança. No momento de uma requisição DNS, o cliente simplesmente que a informação recebida é válida e legítima. Essa forma de requisição cria vulnerabilidades com o envenenamento de \textit{cache}. 

Para se corigir tais vulnerabilidades, foram definidas extensões de segurança para prover confiabilidade nos registros trocados via DNS: o DNSSEC. Para que não fosse modificada a forma como o DNS opera, o DNSSEC simplemente adiciona novos tipos de registros ao DNS como o RRSIG e o DNSKEY e podem ser requisitados da mesma forma como registros do tipo A, CNAME ou MX.

Neste trabalho, pretende-se apresentar como configurar um DNS com as extensões de segurança segundo o DNSSEC e apresentar o passo-a-passo para que um cliente faça requisições seguras. O processo como um todo para configuração das ferramentas utilizadas e como executar o sistema é apresentado a seguir.

\chapter{DNS  - \textit{Domain Name System}}
\label{chap_dns}

Uma consulta DNS comum consiste basicamente de uma requisição enviada pelo cliente seguida de uma resposta devolvida pelo servidor. Nesse sentido, é interessante a utilização de um protocolo de transporte sem estabelecimento de conexão como o UDP para não acarretar em um atraso na requisição, como no caso do protocolo TCP que necessita de um \textit{handshake} de três vias. O processo para configuração do servidor DNS utilizado neste trabalho está descrito a seguir.

\section{Configuração do Servidor DNS}
\label{sec_configuracao}

O servido DNS utilizado nesse experimento foi o Bind9 - \textit{Berkeley Internet Name Domain} - comumente encontrado em distribuições linux.

Para a instalação do servidor Bind, caso não esteja instalado, pode-se instalar seguindo o seguinte passo:

\begin{lstlisting}[language=bash]

	sudo apt-get install bind9 dnsutils

\end{lstlisting}

Após a instalação do servidor DNS é necessario configurar as zonas de endereçamento. Cada zona consiste em um mapeamento de uma URL para um endereço de IP e vice-versa. Para configurar uma zona no Bind, deve-se criar uma arquivo de zona.

Navegue até a pasta referente ao servidor Bind e crie uma pasta para armazenar os arquivos de zona.

\begin{lstlisting}[language=bash]

	cd /etc/bind

	mkdir -p zones/master

\end{lstlisting}

O arquivo de zona criado é referênte ao site www.tcpdump.org. Portando o nome do arquivo deve ser: \textit{db.tcpdump.org} e deve conter o seguinte conteudo:

\begin{lstlisting}[language=bash]

;
; BIND data file for tcpdump.org
;
$TTL    3h
@       IN      SOA     ns1.tcpdump.org. admin.tcpdump.org. (
                          1        ; Serial
                          3h       ; Refresh after 3 hours
                          1h       ; Retry after 1 hour
                          1w       ; Expire after 1 week
                          1h )     ; Negative caching TTL of 1 day
;
@       IN      NS      ns1.tcpdump.org.
@       IN      NS      ns2.tcpdump.org.


tcpdump.org.    IN      MX      10      mail.tcpdump.org.
tcpdump.org.    IN      A       192.139.46.66
ns1                     IN      A       192.139.46.66
ns2                     IN      A       192.139.46.66
www                     IN      CNAME   tcpdump.org.
mail                    IN      A       192.139.46.66
ftp                     IN      CNAME   tcpdump.org.

\end{lstlisting}

Com o arquivo de zona configurado é necessário informar ao ao servidor DNS a localização desse arquivo.

\begin{lstlisting}[language=bash]
zone "tcpdump.org" {
       type master;
       file "/etc/bind/zones/master/db.linuxconfig.org";
};
\end{lstlisting}

Com esses arquivos é possivel a partir de uma URL descobrir seu IP.

Uma última coisa é adicionar um endereço de IP de um servidor DNS estável no arquivo named.conf.options. Este endereço de IP é usado quando o servidor DNS local não sabe a resposta para a consulta de resolução de nome. Para este experimento foi utilizado o servidor de DNS do google - 8.8.8.8 ou 8.8.4.4.

\begin{lstlisting}[language=bash]
 forwarders {
     8.8.4.4;
};
\end{lstlisting}

Com as configurações realizadas liga-se o servidor Bind da seguinte forma:

\begin{lstlisting}[language=bash]
 cd /etc/init.d/
 sudo ./bind9 start
\end{lstlisting}

Com os servidor DNS Bind é possivel obter o endereço de IP de um site através de um requisição, utilizando o comando dig, no servidor DNS local. Nesse experimento o IP da máquina com o servidor DNS é: 192.168.1.23

\begin{lstlisting}[language=bash]
 dig @192.168.1.23 tcpdump.org

 ; <<>> DiG 9.8.3-P1 <<>> @192.168.1.23 tcpdump.org
; (1 server found)
;; global options: +cmd
;; Got answer:
;; ->>HEADER<<- opcode: QUERY, status: NOERROR, id: 30817
;; flags: qr aa rd ra; QUERY: 1, ANSWER: 1, AUTHORITY: 2, ADDITIONAL: 2

;; QUESTION SECTION:
;tcpdump.org.			IN	A

;; ANSWER SECTION:
tcpdump.org.		10800	IN	A	192.139.46.66

;; AUTHORITY SECTION:
tcpdump.org.		10800	IN	NS	ns1.tcpdump.org.
tcpdump.org.		10800	IN	NS	ns2.tcpdump.org.

;; ADDITIONAL SECTION:
ns1.tcpdump.org.	10800	IN	A	192.139.46.66
ns2.tcpdump.org.	10800	IN	A	192.139.46.66

;; Query time: 165 msec
;; SERVER: 192.168.1.23#53(192.168.1.23)
;; WHEN: Sat Nov 14 16:18:10 2015
;; MSG SIZE  rcvd: 113
\end{lstlisting}

\section{Vulnerabilidades}
\label{sec_vulnerabilidades}

Com a utilização de protocolo UDP, como não há estabelecimento de conexão apenas um QID (i.e. \textit{query ID}), um atacante pode tentar utilizar vários pacotes para simular respostas de um servidor DNS conhecido e eventualmente enviar um pacote com um QID válido para o cliente. Nesse momento, o atacante conseguirá inserir no \textit{cache} do cliente um redirecionamento de uma página conhecida para uma outra controlada por ele. Essa é apenas uma vulnerabilidade conhecida como envenenamento de \textit{cahe}. Esse e outros tipos de ataques são melhor explicados a seguir.

{\bf Envenenamento de Cache (Cache Poisoning): }este é um dos mais conhecidos e difundidos ataques a segurança do DNS, ocorrendo quando um cliente realiza uma consulta para um determinado dominio, esta consulta é feita a uma zona DNS, porém esta zona não é fornecida por um servidor autoritativo os quais possuem os registros originais que associam aquele domínio a seu endereço de IP, assim um outro servidor de Cache DNS que encontra-se configurado nos resolvers do computador do cliente responde primeiro a requisição do cliente, levando o este a um servidor que não é o servidor de origem do dominio, assim fornecendo informações forjadas ao cliente. Este cenario é comumente utilizado para levar o utilizador as denomidas páginas de Phishing, onde os dados submetidos pelo resolver são capturadas pelo hacker.

{\bf Ataque de negação de serviço ou DDoS (Distributed Denial of Service): }Acontece quando um invasor tenta negar a disponibilidade de serviços DNS, atraves de consultas recursivas, consumindo todos os resursos como memoria e processador do servidor. Iso ocorre devido ao alto acesso ao servidor ocasionando um uso execivo de processamento, tornando o mesmo indisponivel.

{\bf Modificação de dados: } Apos ocupar uma rede usando DNS, O invasor tenta usar os endereços IP válidos, em pacotes IP criados pelo invasor, assim esses pacotes passam a ter a aparência de um endereço IP válido na rede. Normalmente isso é denominado falsificação de IP.

\chapter{ DNSSEC  - \textit{Domain Name System Security Extensions}}
\label{chap_dnssec}

O DNSSEC é uma solução que expande a segurança do serviço DNS, permitindo validar a origem dos dados, assegurar que a informação recebida de um servidor DNS é autêntica, confirmando que a informação não foi alterada durante a passagem pelos vários “nodos” da Internet e confirmar a inexistência de um domínio.

O sistema DNSSEC introduz apenas alguns novos tipos de Registos nas zonas DNS, nomeadamente DNSKEY, RRSIG e DS, sendo compatível com servidores ou clientes que não tenham implementado o DNSSEC. As extensões de segurança DNSSEC utilizam criptografia assimétrica, permitindo que a verificação e troca de informação DNS entre o servidor e o cliente seja privada e entregue sem alterações, recorrendo a uma assinatura digital e um conjunto de chaves privadas e publicas.

O DNSSEC é baseado numa cadeia de confiança, onde inicialmente é validada a raiz “.”, assinada digitalmente pela ICANN e IANA. Posteriormente a raiz assina a zona “.pt” utilizando a sua chave privada e anexa-lhe a chave publica, permitindo a partir desse momento que quando um servidor de nomes local fizer uma consulta relacionada com um domínio “.pt”, este possa validar a autenticidade da raiz “.pt” recorrendo à chave publica. O processo é repetido, até chegar à zona final correspondente ao domínio que está a ser assinado e validado. As extensões de segurança do DNS são totalmente compatíveis com IPv6, uma vez que apenas é validada a integridade dos registos.

\section{Configuração do Servidor DNS Seguro}
\label{sec_config_segura}

Passos para a configuração segura

Exemplo de código
\begin{lstlisting}[language=bash]

	sudo aptitude install tshark

	sudo aptitude install texlive

	sudo aptitude install texlive-lang-portuguese

\end{lstlisting}

\chapter{Resultados}

\section{Implementação da Solução}
O ecosistema desenvolvido nesse projeto, tem como base a utilização de máquinas \textit{Vagrant}, que provê serviços de virtualização de ambientes para facilitar configuração de diferentes ambientes para vários sistemas operacionais. O provedor de máquinas virtuais utilizado neste trabalho foi o VirtualBox. Para a configuração do ambiente, é necessário a instalação do vagrant \footnote{Vagrant disponível em: https://www.vagrantup.com/} e do VirtualBox \footnote{VirtualBox disponível em: https://www.virtualbox.org/}.

Foram desenvolvidos \textit{scripts} para inicialização das máquinas e configurações dos serviços, todos os passos para subir o ambiente completo foi passado para o script executável \textit{quick-start} que sobe as máquinas e gera um arquivo de \textit{log} para uma requisição a um serviço DNS seguro provido pelo servidor DNS configurado. Todo o ambiente pode ser desligado por meio do \textit{script} de \textit{quick-exit} que destrói os ambientes criados e remove arquivos temporários.

Execute os seguintes passos para subir o ambiente: Com as dependências instaladas, execute o arquivo quick-start encontrado na pasta vagrant. Esse executável irá inicializar um servidor DNS seguro configurando a zona \textit{redesfga.com}, uma servidor apache para onde a zona configurada aponta e um cliente que realizará uma requisição de resolução de nome ao servidor DNS. O resultado da requisição é encontrado no arquivo dnssec.log na pasta vagrant, e mostra todo o processo executado para fazer a requisição segura do domínio passado.

É importante ressaltar que o DNSSEC configurado não provê registro de \textit{delegation signer} (DS), isso devido ao custo atrelado a aquisição de um registro em servidores autorizados. Entretanto, o tópico a seguir mostra como configurar DS para um servidor DNS seguro.

O processo inteiro para configuração pode ser encontrado na \textit{wiki} \footnote{documentação sobre configuração do DS: https://github.com/ArthurJahn/FRC-Final/wiki.}  
\section{Considerações}

O DNS é um serviço crucial para o funcionamento da internet atualmente. Entretanto não foi idealizado provendo serviços de segurança, o que gera várias vulnerabilidades que são utilizadas para atacantes gerarem comportamentos idesejados da rede. 

Com esse trabalho foi possível identificar vulnerabilidades do serviço de DNS, configurar extensões de segurança DNSSEC, verificar os paços para a aquisição de respostas de requisições DNS de forma segura e compreender como os serviços de verificação de autenticidade da internet funcionam e quem é capaz de prover tais serviços. 

Nesse sentido todos os objetivos propostos foram alcançados e o aprendizado obtido por meio da atividade prática mostrou a complexidade das operações relacionadas ao provimento de serviços seguros na internet.

\end{document}  %Fim do documento
