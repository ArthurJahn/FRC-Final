\documentclass[a4paper]{report} %padrao letterpaper, 10pt
\usepackage[utf8]{inputenc}
\usepackage[brazil]{babel}
\usepackage{amsfonts,amssymb,graphicx,enumerate}
\usepackage[centertags]{amsmath}
% Configuracoes de pagina
\usepackage[lmargin=3cm,rmargin=3cm,tmargin=3cm,bmargin=3cm]{geometry}
% Layout da pagina
\usepackage{hyperref}
% Layout para código
\usepackage{listings}
% Font for code style
\usepackage{courier}
% The 68 standard colors known to dvips
\usepackage[usenames,dvipsnames]{xcolor}
\hypersetup{pdfpagelayout=SinglePage, % ou TwoPageLeft
    colorlinks=true,
    pdftitle={Relatório de asuhaush Redes},
    pdfauthor={Áulus Diniz, Arthur Jahn, Gabriel Araújo, Jonatas Lenon}}

%*******************************************************
\definecolor{light-gray}{gray}{0.90}
\lstset{
  backgroundcolor=\color{light-gray},
  basicstyle=\footnotesize\ttfamily,
  breaklines=true
}

\title{Fundamentos de Redes de Computadores: \\ Servidor DNS com Extensões DNSSEC}
\author{Arthur Jahn, Áulus Diniz, Gabriel Araújo, Jonatas Lenon}
\date{15 de Novembro de 2015}    %para ocultar a data digite: \date{ }
%*******************************************************
\begin{document}    %Inicio do documento
\maketitle  %cria o titulo na capa

\tableofcontents %Sumario
%-------------------------------------------------------
\chapter{ DNS  - \textit{Domain Name System}}
\label{chap_dns}

Explicar o serviço, protocolos, e afins

\section{Configuração do Servidor DNS}
\label{sec_configuracao}

Exemplo de código
\begin{lstlisting}[language=bash]

	sudo aptitude install tshark

	sudo aptitude install texlive

	sudo aptitude install texlive-lang-portuguese

\end{lstlisting}

\section{Vulnerabilidades }
\label{sec_vulnerabilidades}

Explicar as vulnerabilidades e exemplificar no servidor.

\chapter{ DNSSEC  - \textit{Domain Name System Security Extensions}}
\label{chap_dnssec}

Explicar as modificações no serviço

\section{Configuração do Servidor DNS Seguro}
\label{sec_config_segura}

Passos para a configuração segura

Exemplo de código
\begin{lstlisting}[language=bash]

	sudo aptitude install tshark

	sudo aptitude install texlive

	sudo aptitude install texlive-lang-portuguese

\end{lstlisting}

\section{Soluções para Vulnerabilidades }
\label{sec_solucoes}

Explicar as vulnerabilidades e exemplificar no servidor a configuração da solução.


% Referencias bibliograficas
\begin{thebibliography}{99}
\end{thebibliography}

\addcontentsline{toc}{chapter}{Referências Bibliográficas}

\end{document}  %Fim do documento
