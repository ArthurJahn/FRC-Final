\documentclass[a4paper]{report} %padrao letterpaper, 10pt
\usepackage[utf8]{inputenc}
\usepackage[brazil]{babel}
\usepackage{amsfonts,amssymb,graphicx,enumerate}
\usepackage[centertags]{amsmath}
% Configuracoes de pagina
\usepackage[lmargin=3cm,rmargin=3cm,tmargin=3cm,bmargin=3cm]{geometry}
% Layout da pagina
\usepackage{hyperref}
% Layout para código
\usepackage{listings}
% Font for code style
\usepackage{courier}
% The 68 standard colors known to dvips
\usepackage[usenames,dvipsnames]{xcolor}
\hypersetup{pdfpagelayout=SinglePage, % ou TwoPageLeft
    colorlinks=true,
    pdftitle={Relatório de asuhaush Redes},
    pdfauthor={Áulus Diniz, Arthur Jahn, Gabriel Araújo, Jonatas Lenon}}

%*******************************************************
\definecolor{light-gray}{gray}{0.90}
\lstset{
  backgroundcolor=\color{light-gray},
  basicstyle=\footnotesize\ttfamily,
  breaklines=true
}

\title{Fundamentos de Redes de Computadores: \\ Servidor DNS com Extensões DNSSEC}
\author{Arthur Jahn, Áulus Diniz, Gabriel Araújo, Jonatas Lenon}
\date{15 de Novembro de 2015}    %para ocultar a data digite: \date{ }
%*******************************************************
\begin{document}    %Inicio do documento
\maketitle  %cria o titulo na capa

\tableofcontents %Sumario
%-------------------------------------------------------
\chapter{Problema e Motivação}
\label{chap_problema}

Aqui devem ser apresentados os problemas de segurança do DNS e como eles afetam os serviços disponibilizados pela internet.

Deve mostrar também o porquê da utilização de DNSSEC, as suas vantagens e o escopo de segurança (ou seja, quais tipos de ataques são evitados com a utilização de um DNSSEC e quais não).

\chapter{DNS  - \textit{Domain Name System}}
\label{chap_dns}

O DNS provê serviços de resolução de \textit{hosts} por meio da tradução de nomes de domínios em números de endereçamento IP. Como um mesmo domínio pode estar vinculado a vários endereços IP, esse serviço pode ser responsável também por distribuição de carga entre os \textit{hosts} de destino, alternando o endereço fornecido ao cliente quando uma solicitação de resolução de domínio é feita.

Uma consulta DNS consiste basicamente de uma requisição enviada pelo cliente seguida de uma resposta devolvida pelo servidor. Nesse sentido, é interessante a utilização de um protocolo de transporte sem estabelecimento de conexão como o UDP para não acarretar em um atraso na requisição, como no caso do protocolo TCP que necessita de um \textit{handshake} de três vias.

O serviço de DNS utiliza normalmente a porta 53.

\section{Configuração do Servidor DNS}
\label{sec_configuracao}

O servido DNS utilizado nesse experimento foi o Bind9 - \textit{Berkeley Internet Name Domain} - comumente encontrado em distribuições linux.

Para a instalação do servidor Bind, caso não esteja instalado, pode-se instalar seguindo o seguinte passo:

\begin{lstlisting}[language=bash]

	sudo apt-get install bind9 dnsutils

\end{lstlisting}

Após a instalação do servidor DNS é necessario configurar as zonas de endereçamento. Cada zona consiste em um mapeamento de uma URL para um endereço de IP e vice-versa. Para configurar uma zona no Bind, deve-se criar uma arquivo de zona.

Navegue até a pasta referente ao servidor Bind e crie uma pasta para armazenar os arquivos de zona.

\begin{lstlisting}[language=bash]

	cd /etc/bind

	mkdir -p zones/master

\end{lstlisting}

O arquivo de zona criado é referênte ao site www.tcpdump.org. Portando o nome do arquivo deve ser: \texdb{db.tcpdump.org} e deve conter o seguinte conteudo:

\begin{lstlisting}[language=bash]

;
; BIND data file for tcpdump.org
;
$TTL    3h
@       IN      SOA     ns1.tcpdump.org. admin.tcpdump.org. (
                          1        ; Serial
                          3h       ; Refresh after 3 hours
                          1h       ; Retry after 1 hour
                          1w       ; Expire after 1 week
                          1h )     ; Negative caching TTL of 1 day
;
@       IN      NS      ns1.tcpdump.org.
@       IN      NS      ns2.tcpdump.org.


tcpdump.org.    IN      MX      10      mail.tcpdump.org.
tcpdump.org.    IN      A       192.139.46.66
ns1                     IN      A       192.139.46.66
ns2                     IN      A       192.139.46.66
www                     IN      CNAME   tcpdump.org.
mail                    IN      A       192.139.46.66
ftp                     IN      CNAME   tcpdump.org.

\end{lstlisting}

Com o arquivo de zona configurado é necessário informar ao ao servidor DNS a localização desse arquivo.

\begin{lstlisting}[language=bash]
zone "tcpdump.org" {
       type master;
       file "/etc/bind/zones/master/db.linuxconfig.org";
};
\end{lstlisting}

Com esses arquivos é possivel a partir de uma URL descobrir seu IP.

Uma última coisa é adicionar um endereço de IP de um servidor DNS estável no arquivo named.conf.options. Este endereço de IP é usado quando o servidor DNS local não sabe a resposta para a consulta de resolução de nome. Para este experimento foi utilizado o servidor de DNS do google - 8.8.8.8 ou 8.8.4.4.

\begin{lstlisting}[language=bash]
 forwarders {
     8.8.4.4;
};
\end{lstlisting}

Com as configurações realizadas liga-se o servidor Bind da seguinte forma:

\begin{lstlisting}[language=bash]
 cd /etc/init.d/
 sudo ./bind9 start
\end{lstlisting}

Com os servidor DNS Bind é possivel obter o endereço de IP de um site através de um requisição, utilizando o comando dig, no servidor DNS local. Nesse experimento o IP da máquina com o servidor DNS é: 192.168.1.23

\begin{lstlisting}[language=bash]
 dig @192.168.1.23 tcpdump.org

 ; <<>> DiG 9.8.3-P1 <<>> @192.168.1.23 tcpdump.org
; (1 server found)
;; global options: +cmd
;; Got answer:
;; ->>HEADER<<- opcode: QUERY, status: NOERROR, id: 30817
;; flags: qr aa rd ra; QUERY: 1, ANSWER: 1, AUTHORITY: 2, ADDITIONAL: 2

;; QUESTION SECTION:
;tcpdump.org.			IN	A

;; ANSWER SECTION:
tcpdump.org.		10800	IN	A	192.139.46.66

;; AUTHORITY SECTION:
tcpdump.org.		10800	IN	NS	ns1.tcpdump.org.
tcpdump.org.		10800	IN	NS	ns2.tcpdump.org.

;; ADDITIONAL SECTION:
ns1.tcpdump.org.	10800	IN	A	192.139.46.66
ns2.tcpdump.org.	10800	IN	A	192.139.46.66

;; Query time: 165 msec
;; SERVER: 192.168.1.23#53(192.168.1.23)
;; WHEN: Sat Nov 14 16:18:10 2015
;; MSG SIZE  rcvd: 113
\end{lstlisting}

\section{Vulnerabilidades}
\label{sec_vulnerabilidades}

Explicar as vulnerabilidades e exemplificar no servidor.

\chapter{ DNSSEC  - \textit{Domain Name System Security Extensions}}
\label{chap_dnssec}

Explicar as modificações no serviço

\section{Configuração do Servidor DNS Seguro}
\label{sec_config_segura}

Passos para a configuração segura

Exemplo de código
\begin{lstlisting}[language=bash]

	sudo aptitude install tshark

	sudo aptitude install texlive

	sudo aptitude install texlive-lang-portuguese

\end{lstlisting}

\section{Soluções para Vulnerabilidades }
\label{sec_solucoes}

Explicar as vulnerabilidades e exemplificar no servidor a configuração da solução.

\chapter{Resultados}

\section{Implementação da Solução}

Aqui devem ser descritos os paços para a construção do estudo sobre o DNSSEC. Deve ser apresentado o esquema das máquinas vagrant criadas e como o cliente faz as requisições.

Deve ser mostrado o processo para reprodução do nosso estudo (por meio do script de quick-start no link do github). E também o processo que falta para se implementar o \textit{delegation signer} (DS). 

\section{Considerações}

Aqui vão vir as considerações sobre os resultados obtidos e a adequação com o propósito do trabalho.

% Referencias bibliograficas
\begin{thebibliography}{99}
\end{thebibliography}

\addcontentsline{toc}{chapter}{Referências Bibliográficas}

\end{document}  %Fim do documento
